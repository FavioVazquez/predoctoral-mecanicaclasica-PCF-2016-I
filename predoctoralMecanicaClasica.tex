\documentclass[a4paper,10pt]{article}
\usepackage[utf8]{inputenc}
\usepackage[spanish]{babel}
\usepackage[affil-it]{authblk}
\usepackage{enumerate}
\usepackage{graphicx}
\usepackage{hyperref}
\usepackage{amsmath}
\usepackage{amssymb}
\usepackage{cancel}
\usepackage[usenames, dvipsnames]{color}
\usepackage{tikz}
\usepackage[labelfont=bf]{caption}
\usepackage{subcaption} %Multiple images
\usepackage{multicol} % Multiple columns
\usepackage{float}
\usepackage{cleveref}
 \usepackage{relsize} % bigger math symbols
\usepackage[margin=1.4in]{geometry}
\usepackage[titletoc,toc,title]{appendix}
\usepackage{enumitem}
\usepackage{etoolbox}
\usepackage{mdframed} %frame theorems
\usetikzlibrary{calc}
\numberwithin{equation}{section}

% Enviroment for theorems
\newmdtheoremenv[frametitle=Teorema]{theo}{Theorem}

% Circled words
\newcommand{\circled}[2][]{%
  \tikz[baseline=(char.base)]{%
    \node[shape = circle, draw, inner sep = 1pt]
    (char) {\phantom{\ifblank{#1}{#2}{#1}}};%
    \node at (char.center) {\makebox[0pt][c]{#2}};}}
\robustify{\circled}

%Appendices in spanish
\renewcommand{\appendixname}{Ap\'endices}
\renewcommand{\appendixtocname}{Ap\'endices}
\renewcommand{\appendixpagename}{Ap\'endices}

%Zero delimiter
\newcommand{\zerodel}{.\kern-\nulldelimiterspace}

%Columns separation
\setlength{\columnsep}{1cm}

%Indentation
\setlength{\parindent}{0ex}

%Multiple References

\crefrangelabelformat{equation}{(#3#1#4--#5\crefstripprefix{#1}{#2}#6)}

\usepackage{xparse}

%Boxes

\newcommand*{\boxcolor}{blue}
\makeatletter
\renewcommand{\boxed}[1]{\textcolor{\boxcolor}{%
\tikz[baseline={([yshift=-1ex]current bounding box.center)}] \node [rectangle, minimum width=1ex,rounded corners,draw] {\normalcolor\m@th$\displaystyle#1$};}}
 \makeatother

%Constantes
\newcommand{\euler}{\mathrm{e}}
\newcommand{\im}{i}

%Lemas, teoremas, definiciones y pruebas
\newcommand{\definicion}{\textbf{Definición: }}
\newcommand{\lema}{\textbf{Lema: }}
\newcommand{\teorema}{\textbf{Teorema: }}
\newcommand{\prueba}{\textbf{Prueba: }}
\newcommand{\proposicion}{\textbf{Proposición: }}
\newcommand{\corolario}{\textbf{Corolario: }}

% Definición de las secciones y su numeración

\makeatletter
\def\@seccntformat#1{%
  \expandafter\ifx\csname c@#1\endcsname\c@section\else
  \csname the#1\endcsname\quad
  \fi}
\makeatother

%opening
\title{Examen Predoctoral de Mecánica Clásica. \\
Semestre 2016-I.}
\author{Favio Vázquez\thanks{Correo: favio.vazquezp@gmail.com}}\affil{Instituto de Ciencias Nucleares. Universidad Nacional Autónoma de México.}
\date{}

\begin{document}

\makeatletter
\def\@maketitle{%
  \newpage
  \null
  \vskip 2em%
  \begin{center}%
  \let \footnote \thanks
    {\Large\bfseries \@title \par}%
    \vskip 1.5em%
    {\normalsize
      \lineskip .5em%
      \begin{tabular}[t]{c}%
        \@author
      \end{tabular}\par}%
    \vskip 1em%
    {\normalsize \@date}%
  \end{center}%
  \par
  \vskip 1.5em}
\makeatother

\maketitle

\section{Preguntas teóricas}

\subsection{Pregunta teórica 1}

Sobre las formulaciones de la mecánica, discuta los siguientes puntos:

\begin{enumerate}[label=\alph*)]
 \item Bajo qué condiciones las ecuaciones de Euler-Lagrange determinan todas las aceleraciones del
 sistema.
 \item En el caso Hamiltoniano, cuál es la condición equivalente.
 \item Dentro del formalismo de Hamilton-Jacobi qué garantiza el que la funcional generadora
 proporcione una solución de las ecuaciones de Hamilton.
\end{enumerate}

\vspace{.3cm}

\underline{Solución:} \vspace{.3cm}

\begin{enumerate}[label=\alph*)]
 \item R:
\end{enumerate}

La formulación lagrangiana en coordenadas generalizadas es equivalente a la formulación 
newtoniana, por lo tanto la ley de la naturaleza se expresa como una ecuación diferencial 
de segundo orden. Por lo tanto, para poder determinar completamente el estado 
de un sistema mecánico, y definir unívocamente todas las aceleraciones del mismo 
requerimos conocer simultáneamente las coordenadas y las velocidades en un instante 
dado. Entonces ya que las ecuaciones de Euler-Lagrange, contienen derivadas 
sobre la lagrangiana que depende de las coordenadas, las velocidades y 
posiblemente el tiempo, al conocer simultáneamente las coordenadas y velocidades 
en un instante dado, podremos determinar todas las aceleraciones del sistema.

\begin{enumerate}[label=\alph*)]
  \setcounter{enumi}{1}
 \item R:
\end{enumerate}

En el caso de la formulación hamiltoniana, ya que es equivalente a la formulación 
lagrangiana y por lo tanto a la newtoniana, se cumplen las mismas condiciones. Ahora 
en este caso, debido a que las ecuaciones de Hamilton contemplan derivadas de la 
hamiltoniana del sistema, y ésta depende de coordenadas e impulsos, se deben conocer 
ahora en un instante dado simultáneamente las coordenadas y los impulsos del sistema 
mecánico para determinar todas las aceleraciones del mismo. Debido a la relación 
directa entre los impulsos y las velocidades en esta formulación, esta condición 
se puede considerar idéntica a la anterior, aunque es un poco más formal hablar de 
impulsos en este caso, ya que la formulación hamiltoniana existe en el ámbito del 
espacio fase de coordenadas e impulsos.

\begin{enumerate}[label=\alph*)]
  \setcounter{enumi}{3}
 \item R:
\end{enumerate}

Lo que nos garantiza, en la formulación de Hamilton-Jacobi, que la funcional generadora 
proporcione una solución de las ecuaciones de Hamilton, es que ésta ea una integral 
completa, es decir que sea una solución de la ecuación en derivadas parciales resultante 
en esta formulación, que contenga tantas constantes arbitrarias independientes como 
variables independientes existan. Y debido a que en la ecuación de Hamilton-Jacobi 
las variables independientes son las coordenadas y el tiempo, para un sistema de 
$n$ grados de libertad, una integral completa debe contener $n+1$ constantes arbitrarias. 
Estas constantes que se deben obtener en el ámbito de una integral completa, nos permitirán 
expresar a las coordenadas y los impulsos como funciones del tiempo, y las constantes, 
con lo cual se puede demostrar que estas variables cumplirán con ecuaciones 
diferenciales que tienen forma hamiltoniana.

\subsection{Pregunta Teórica 2}

Considere un cuerpo rígido con momentos de inercia $I_1 > I_2 > I_3$. Si sobre el cuerpo no se 
ejercen torcas, ¿qué ejes del cuerpo son estables e inestables bajo pequeñas perturbaciones y 
por qué?

\vspace{.3cm}

\underline{Solución:} \vspace{.3cm} 

Debemos ver en qué casos, con respecto a la dirección de la velocidad angular, el 
movimiento será estable y con respecto a qué eje principal. Partimos de las ecuaciones 
de Euler para un cuerpo rígido al cual no se le aplican torcas,

\begin{align}
 \label{eq:pregunta2eq1}
 I_1\dot{\omega}_1 - \omega_2\omega_3(I_2 - I_3) = 0, \\
 \label{eq:pregunta2eq2}
 I_2\dot{\omega}_2 - \omega_3\omega_1(I_3 - I_1) = 0, \\
 \label{eq:pregunta2eq3}
 I_3\dot{\omega}_3 - \omega_1\omega_2(I_1 - I_2) = 0.
\end{align}

Estas ecuaciones nos permitirán estudiar, cuando el cuerpo esté en movimiento, qué 
condiciones deben cumplirse para que el movimiento sea estable.

\vspace{.3cm}

Si suponemos que la velocidad angular es en dirección al eje $x$, entonces se 
cumplirá que $\omega_1 \gg \omega_2,\omega_3$. Ahora debido a que $\omega_2$ y 
$\omega_3$ se mantendrán pequeños con respecto a $\omega_1$, el movimiento será 
estable y debido a que no hay torque $|\overrightarrow{\omega}| = \text{cte}$, y 
debido a que $\omega = \sqrt{\omega_1^2 + \omega_2^2 + \omega_3^2}$, tendremos que 
$\omega_2^2 + \omega_3^2 \ll \omega_1^2$, y entonces $\omega = \sqrt{\omega_1^2} = 
\omega_1$, y entonces podemos tomar a $\omega_1$ como constante, al menos 
a primer orden.

Tomando la derivada temporal de \eqref{eq:pregunta2eq2} 

\begin{equation}
 \frac{d}{dt}\left[I_2\dot{\omega}_2 - \omega_3\omega_1(I_3 - I_1) \right] = 0,
\end{equation}

y debido a que $\omega_1$ es constante, 

\begin{equation}
 I_2\ddot{\omega}_2 - \dot{\omega_3}\omega_1(I_3 - I_1) = 0,
\end{equation}

\begin{equation}
 \therefore \ddot{\omega}_2 = \frac{\dot{\omega}_3\omega_1(I_3-I_1)}{I_2}.
  \label{eq:pregunta2eq4}
\end{equation}

Sustituyendo $\dot{\omega}_3$ de \eqref{eq:pregunta2eq3}, 

\begin{equation}
 \ddot{\omega_2} = \frac{\omega_2\omega_1^2(I_3 - I_1)(I_1 - I_2)}{I_2I_3}.
 \label{eq:pregunta2eq5}
\end{equation}

Tomando la derivada temporal de \eqref{eq:pregunta2eq3} 

\begin{equation}
 \frac{d}{dt}\left[I_3\dot{\omega}_3 - \omega_1\omega_2(I_1 - I_2) \right] = 0,
\end{equation}

\begin{equation}
 I_3\ddot{\omega}_3 - dot{\omega}_2\omega_1(I_1 - I_2) = 0,
\end{equation}

\begin{equation}
 \therefore \ddot{\omega}_3 = \frac{\dot{\omega}_2\omega_1(I_1-I_2)}{I_3}.
 \label{eq:pregunta2eq6}
\end{equation}

Sustituyendo $\dot{\omega}_2$ de \eqref{eq:pregunta2eq2}, en \eqref{eq:pregunta2eq6}

\begin{equation}
 \ddot{\omega_3} = \frac{\omega_3\omega_1^2(I_1 - I_2)(I_3 - I_1)}{I_3I_2}.
 \label{eq:pregunta2eq7}
\end{equation}

Ahora como $I_1 > I_2,I_3$, los lados derechos de \eqref{eq:pregunta2eq5} y 
\eqref{eq:pregunta2eq7} serán negativos y por lo tanto tendremos un equilibrio 
estable, con un movimiento tipo oscilador armónico, y $\omega_2$ y $\omega_3$ 
oscilarán al rededor del punto de equilibrio. Vemos entonces que un movimiento 
en dirección del eje $x$ es estable, y por consiguiente el momento principal 
de inercia $I_1$ será estable. 

\vspace{.3cm}

Supongamos ahora que el movimiento es en dirección al eje $z$, utilizando los 
mismos argumentos, vemos que debido a que $\omega_3 \gg \omega_1,\omega_2$, 
$|\omega| = \text{cte}$, entonces $\omega = \omega_3$, y podemos considerar 
que $\omega_3$ es estable, al menos a primer orden. 

Tomando la derivada temporal de \eqref{eq:pregunta2eq1} 

\begin{equation}
 \frac{d}{dt}\left[I_1\dot{\omega}_1 - \omega_2\omega_3(I_2 - I_3) \right] = 0,
\end{equation}

y debido a que $\omega_3$ es constante, 

\begin{equation}
 I_1\ddot{\omega}_1 - \dot{\omega}_2\omega_3(I_2 - I_3) = 0,
\end{equation}

\begin{equation}
 \therefore \ddot{\omega}_1 = \frac{\dot{\omega}_2\omega_3(I_2-I_3)}{I_1}.
  \label{eq:pregunta2eq8}
\end{equation}

Sustituyendo $\dot{\omega}_2$ de \eqref{eq:pregunta2eq2}, 

\begin{equation}
 \ddot{\omega_1} = \frac{\omega_1\omega_3^2(I_2 - I_3)(I_3 - I_1)}{I_2I_1}.
 \label{eq:pregunta2eq9}
\end{equation}

Tomando la derivada temporal de \eqref{eq:pregunta2eq2} 

\begin{equation}
 \frac{d}{dt}\left[I_2\dot{\omega}_2 - \omega_3\omega_1(I_3 - I_1) \right] = 0,
\end{equation}

\begin{equation}
 I_2\ddot{\omega}_2 - \dot{\omega}_1\omega_3(I_3 - I_1) = 0,
\end{equation}

\begin{equation}
 \therefore \ddot{\omega}_3 = \frac{\dot{\omega}_1\omega_3(I_3-I_1)}{I_2}.
 \label{eq:pregunta2eq10}
\end{equation}

Sustituyendo $\dot{\omega}_1$ de \eqref{eq:pregunta2eq1}, en \eqref{eq:pregunta2eq10}

\begin{equation}
 \ddot{\omega_2} = \frac{\omega_2\omega_3^2(I_3 - I_1)(I_2- I_3)}{I_1I_2}.
 \label{eq:pregunta2eq11}
\end{equation}

De nuevo como $I_1 > I_2,I_3$, los lados derechos de \eqref{eq:pregunta2eq9} y 
\eqref{eq:pregunta2eq11} serán negativos y por lo tanto tendremos un equilibrio 
estable, con un movimiento tipo oscilador armónico, y $\omega_1$ y $\omega_2$ 
oscilarán al rededor del punto de equilibrio. Vemos entonces que un movimiento 
en dirección del eje $z$ es estable, y por consiguiente el momento principal 
de inercia $I_3$ será estable. 

\vspace{.3cm}

Por último, supongamos que el movimiento es en dirección al eje $y$,  y utilizando los 
mismos argumentos, vemos que debido a que $\omega_2 \gg \omega_1,\omega_3$, 
$|\omega| = \text{cte}$, entonces $\omega = \omega_2$, y podemos considerar 
que $\omega_2$ es estable, al menos a primer orden. 

Tomando la derivada temporal de \eqref{eq:pregunta2eq1} 

\begin{equation}
 \frac{d}{dt}\left[I_1\dot{\omega}_1 - \omega_2\omega_3(I_2 - I_3) \right] = 0,
\end{equation}

y debido a que $\omega_2$ es constante, 

\begin{equation}
 I_1\ddot{\omega}_1 - \dot{\omega}_3\omega_2(I_2 - I_3) = 0,
\end{equation}

\begin{equation}
 \therefore \ddot{\omega}_1 = \frac{\dot{\omega}_3\omega_2(I_2-I_3)}{I_1}.
  \label{eq:pregunta2eq12}
\end{equation}

Sustituyendo $\dot{\omega}_3$ de \eqref{eq:pregunta2eq3}, 

\begin{equation}
 \ddot{\omega_1} = \frac{\omega_1\omega_2^2(I_1 - I_2)(I_2 - I_3)}{I_3I_1}.
 \label{eq:pregunta2eq13}
\end{equation}

Tomando la derivada temporal de \eqref{eq:pregunta2eq3} 

\begin{equation}
 \frac{d}{dt}\left[I_3\dot{\omega}_3 - \omega_1\omega_2(I_1 - I_2) \right] = 0,
\end{equation}

\begin{equation}
 I_3\ddot{\omega}_3 - \dot{\omega}_1\omega_2(I_1 - I_2) = 0,
\end{equation}

\begin{equation}
 \therefore \ddot{\omega}_3 = \frac{\dot{\omega}_1\omega_2(I_1-I_2)}{I_3}.
 \label{eq:pregunta2eq14}
\end{equation}

Sustituyendo $\dot{\omega}_1$ de \eqref{eq:pregunta2eq1}, en \eqref{eq:pregunta2eq14}

\begin{equation}
 \ddot{\omega_3} = \frac{\omega_3\omega_2^2(I_1 - I_2)(I_2- I_3)}{I_3I_2}.
 \label{eq:pregunta2eq15}
\end{equation}

Como $I_2 > I_3$ y $I_1 > I_2$, los lados derechos de \eqref{eq:pregunta2eq13} y 
\eqref{eq:pregunta2eq15} serán positivos y por lo tanto tendremos un equilibrio 
inestable. Vemos entonces que un movimiento en dirección del eje $y$ es inestable, 
y por consiguiente el momento principal de inercia $I_2$ será estable. 

\vspace{.3cm}

Entonces concluimos que los momentos de inercia estables son el mayor $I_1$ y 
el menor $I_3$, siendo $I_2$ el intermedio, inestable. Esto puede verse fácilmente 
si intentamos lanzar una raqueta al aire al mismo tiempo que le imprimimos una 
rotación, ya sea en torno al eje definido por el mango, al eje perpendicular a la 
pala o al perpendicular al mango contenido en la pala; en los primeros casos es 
fácil hacerlo sin producir fuertes bamboleos, en el tercero es prácticamente 
imposible (que es el eje correspondiente al momento principal de inercia intermedio).

\subsection{Pregunta 3}

Sobre un piso sin fricción, una partícula puntual choca elásticamente con una mancuerna de dos 
modos diferentes mostrados en la figura (considere que las partículas de la mancuerna son también
puntuales y que la barra tiene masa despreciable). ¿Qué cantidades se conservan en cada caso? 
¿En qué caso la rapidez del centro de masa de la mancuerna, después dela colisión, es mayor?

\vspace{.3cm}

AGREGAR FIGURA

\vspace{.3cm}

\section{Problemas}

\subsection{Problema 1}

La interacción clásica entre dos átomos de un gas inerte, cada uno de masa $m$ está dada por el
potencial 

$$
V(r) = - \frac{2A}{r^6} + \frac{B}{r^{12}}
$$

con $A$ y $B$ constantes positivas y $r$ la separación entre los dos átomos, $r = |\overrightarrow{r_1} 
- \overrightarrow{r_2}|$.

\begin{enumerate}[label=\alph*)]
 \item Obtenga la hamiltoniana para el sistema de los dos átomos.
 \item Describa completamente el (los) estado(s) clásico(s) de energía mínima del presente 
 sistema.
 \item Si la energía es un poco mayor que la mínima, ¿Cuáles son las posibles frecuencias 
 del movimiento del sistema?
\end{enumerate}

\vspace{.3cm}

\underline{Solución:} \vspace{.3cm}

\subsection{Problema 2}

Considere el siguiente sistema 

$$
L = \frac{m}{2}\dot{q}^2 - af(t)q
$$

con $f(t)$ una función arbitraria del tiempo pero integrable.

\begin{enumerate}[label=\alph*)]
 \item Considerando que una simetría del sistema es aquella que deja invariantes las 
 ecuaciones de movimiento. ¿Existe alguna simetría asociada a este sistema? Si es así,
 calcule la cantidad conservada correspondiente usando el teorema de Noether.
 \item Muestre que, efectivamente su derivada total con respecto del tiempo es cero.
 \item Construya el Hamiltoniano del sistema y escriba la ecuación de Hamilton-Jacobi 
 correspondiente.
 \item Resuelva la ecuación de hamilton-Jacobi y encuentre la funcional generadora de 
 tipo 2.
 \item Considere que $f(t) = \exp{(-bt)}$ con $b > 0$ y las condiciones iniciales
 $q(0) = \beta$ y $\dot{q}(0) = \rho$. Usando la teoría de Hamilton-Jacobi, encuentre 
 la trayectoria de la partícula.
\end{enumerate}

\vspace{.3cm}

\underline{Solución:} \vspace{.3cm}

\begin{enumerate}[label=\alph*)]
 \item R:
\end{enumerate}

Comencemos por obtener las ecuaciones de movimiento con la lagrangiana dada para 
tener un punto de comparación. Usando las ecuaciones de Lagrange 

\begin{equation}
 \frac{d}{dt}\left(\frac{\partial L}{\partial \dot{q}}\right) 
 - \frac{\partial L}{\partial q}
\end{equation}

\begin{equation}
 m\ddot{q} + af(t) = 0.
\end{equation}

Debido a que la lagrangiana depende explícitamente del tiempo, no existirán integrales 
de movimiento, sino solo constantes de movimiento. Comúnmente las simetrías asociadas 
a constantes de movimiento son un poco menos obvias de encontrar, y en general estas 
no provienen de una simetría en la forma estándar. En este caso podemos ver que 
la siguiente transformación, 

\begin{equation}
 Q = q + s,
\end{equation}

deja invariante a las ecuaciones de movimiento. Para ver esto, sustituyamos esta 
transformación en la lagrangiana original\footnote{$\frac{d}{dt}(Q -s) = \dot{Q}$},

\begin{equation}
 L' = \frac{m}{2}\dot{Q}^2 - af(t)(Q-s),
\end{equation}

y obtengamos las ecuaciones de movimiento desde las ecuaciones de Lagrange

\begin{equation}
 \frac{d}{dt}\left(\frac{\partial L}{\partial \dot{Q}}\right) 
 - \frac{\partial L}{\partial Q},
\end{equation}

que nos dan

\begin{equation}
 m\ddot{Q} + af(t) = 0.
\end{equation}

por lo tanto hemos demostrado que la transformación propuesta deja invariante a las 
ecuaciones de movimiento, y con la definición dada en el enunciado, vemos que es 
una simetría. Debido a que tratamos con una constante y no una integral de movimiento, 
la ecuación para obtener la cantidad conservada con el teorema de Noether será 

\begin{equation}
 I = \left(\left\zerodel\frac{\partial L'}{\partial \dot{Q}}
 \frac{\partial q}{\partial s}\right)\right|_{s=0} - G,
\end{equation}

donde en este caso es fácil ver que $\frac{dG}{dt} = - af(t)$, ya que al sustituir 
y hacer los cálculos de las derivadas, tenemos que 

\begin{equation}
 \boxed{I = m\dot{Q} + a\int f(t) dt,}
\end{equation}

y entonces 

\begin{enumerate}[label=\alph*)]
\setcounter{enumi}{1}
 \item R:
\end{enumerate}

\begin{equation}
 \frac{dI}{dt} = m\ddot{Q} + af(t) = 0,
\end{equation}

que es cero debido a la ecuación de movimiento obtenida.

\begin{enumerate}[label=\alph*)]
\setcounter{enumi}{2}
 \item R:
\end{enumerate}

Para encontrar el hamiltoniano usamos la definición de impulso 

\begin{equation}
 p = \frac{\partial L}{\partial \dot{q}},
\end{equation}

para con esta cantidad, poder expresar la velocidad en término del impulso, y luego 
sustituir su valor en la lagrangiana para obtenerla en términos de las posiciones 
e impulsos, construimos así la hamiltoniana con la ecuación 

\begin{equation}
 H = p\dot{q}(q,p,t) - L(q,p,t).
\end{equation}

Tenemos entonces que 

\begin{equation}
 p = \frac{\partial L}{\partial \dot{q}} = m\dot{q},
\end{equation}

y 

\begin{equation}
 \dot{q} = \frac{p}{m},
\end{equation}

\begin{equation}
 \therefore L = \frac{p^2}{2m} - af(t)q,
\end{equation}

y

\begin{equation}
 H = p\frac{p}{m} - \frac{p^2}{2m} - af(t)q,
\end{equation}

\begin{equation}
 \boxed{\therefore H = \frac{p^2}{2m} - af(t)q.}
\end{equation}

Para construir la ecuación de Hamilton-Jacobi, transformamos los impulsos por 
$\frac{\partial S}{\partial q}$ y hacemos uso de la ecuación 

\begin{equation}
 H\left(q,\frac{\partial S}{\partial q},t\right) = - \frac{\partial S}{\partial t},
\end{equation}

donde $S$ es una función generadora de transformaciones canónicas y una solución 
completa de la ecuación. Entonces 

\begin{equation}
 \boxed{\frac{1}{2m}\left(\frac{\partial S}{\partial q}\right)^2 - af(t)q = 
 - \frac{\partial S}{\partial t}.}
\end{equation}

\begin{enumerate}[label=\alph*)]
\setcounter{enumi}{3}
 \item R:
\end{enumerate}

Debido a que la ecuación de Hamilton-Jacobi es dependiente del tiempo, proponemos 
una solución del tipo 

\begin{equation}
 S(q,t) = A(t)q + B(t),
\end{equation}

entonces la ecuación de Hamilton-Jacobi se transforma en

\begin{equation}
\frac{1}{2m} A^2 - af(t)q + \dot{A}q + \dot{B} = 0,
\end{equation}

\begin{equation}
 \therefore [af(t) + \dot{A}]q + \left[\frac{1}{2m}A^2 + \dot{B} \right] = 0,
\end{equation}

y para que esta ecuación los coeficientes de $q$ y de $1$ deben ser idénticamente 
cero, por lo tanto\footnote{Llamaremos $\int f(t) dt = F(t)$}

\begin{equation}
 \dot{A} + af(t) = 0 \Rightarrow dA = - a\int f(t)dt
\end{equation}

\begin{equation}
 \therefore A = -aF(t) + A_0.
\end{equation}

Y por otra parte, utilizando que $A^2 = a^2F^2 - 2aFA_0 + A_0^2$, 

\begin{equation}
 \dot{B} + \frac{1}{2m}A^2 = 0 \Rightarrow dB = 
 - \frac{1}{2m} \int [a^2F^2 - 2aFA_0 + A_0^2] dt
\end{equation}

Y entonces 

\begin{equation}
 \boxed{S = [-aF(t) + A_0]q  - \frac{1}{2m} \int [a^2F^2 - 2aFA_0] dt - \frac{1}{2m}A_0^2t.}
\end{equation}

Como es de costumbre en la formulación de Hamilton-Jacobi, calculamos el impulso 
con 

\begin{equation}
 \boxed{p = \frac{\partial S}{\partial q} = -aF(t) + A_0,}
\end{equation}

y terminamos de calcular la trayectoria con 

\begin{equation}
 \xi = \frac{\partial S}{\partial A_0} = q + \frac{a}{m}\int F dt - \frac{A_0t}{m},
\end{equation}

por lo tanto 

\begin{equation}
 \boxed{q = \xi - \frac{a}{m}\int F dt + \frac{A_0t}{m}.}
\end{equation}

\begin{enumerate}[label=\alph*)]
\setcounter{enumi}{4}
 \item R:
\end{enumerate}

Si $f(t) = \exp{(-bt)}$, tenemos que $F(t) = - \frac{1}{b}\exp{(-bt)}$, y 

\begin{equation}
 \boxed{p = a\frac{1}{b}\exp{(-bt)} + A_0,}
\end{equation}

\begin{equation}
 \boxed{q = \xi - \frac{a}{mb^2}\exp{(-bt)} + \frac{A_0t}{m}.}
\end{equation}

Para aplicar las condiciones iniciales calculamos primero

\begin{equation}
 \dot{q} = \frac{a}{mb}\exp{(-bt)} + \frac{A_0}{m},
\end{equation}

Aplicando la primera condición inicial $q(0) = \beta$, tenemos que 

\begin{equation}
 \xi - \frac{a}{mb^2} = \beta \Rightarrow \xi = \beta + \frac{a}{mb^2},
\end{equation}

y aplicando la segunda $\dot{q}(0) = \rho$, 

\begin{equation}
\frac{a}{mb} + \frac{A_0}{m} = \rho \Rightarrow A_0 = m\rho - \frac{a}{b}, 
\end{equation}

entonces 

\begin{equation}
 \boxed{q(t) = \beta + \frac{a}{mb^2} - \frac{a}{mb^2}\exp{(-bt)} + 
 \left(\rho - \frac{a}{mb} \right)t.}
\end{equation}

\subsection{Problema 3}

Una partícula de masa $m$ está restringida a moverse en el interior de un riel circular 
de radio $R$. El riel circular está fijado al piso en posición vertical. Un pequeño 
motor hacer girar el riel en torno al eje de simetría vertical con rapidez angular 
constante $\omega$ (ver figura). Considere el cero de energía potencial en el piso y 
sea $\theta$ el ángulo que forma el radio vector de posición de la partícula con el 
eje de rotación.

\begin{enumerate}[label=\alph*)]
 \item Determine el Lagrangiano del sistema con constricción y la ecuación de movimiento 
 en $\theta$ para la partícula.
 \item Para que exista una órbita a $\theta_{eq}= \text{cte}$ y distinta de cero, $\omega$
 tiene que ser mayor que cierta $\omega_0$. Determine $\omega_0$.
\end{enumerate}

\vspace{.3cm}

\underline{Solución:} \vspace{.3cm}

\begin{enumerate}[label=\alph*)]
 \item R:
\end{enumerate}

Debido a que la única fuerza que existe es la gravitacional, el potencial se 
escribiría como (utilizando coordenadas polares)

\begin{equation}
 V = mgz = mgR\cos{\theta},
\end{equation}

ahora debido a que se solicita que el cero de la energía potencial esté en el piso, 
hay que cambiar el punto de referencia para el potencial, y ya que al potencial 
gravitacional solo le importa la diferencia entre el punto de referencia y el suelo, 
para acomodar esta condición hacemos primero,

\begin{equation}
 V = mg(z - z_0),
\end{equation}

donde el $z_0$ en este caso será $-R$ ya que el eje $z$ se ha tomando como positivo 
hacia arriba, entonces 

\begin{equation}
 V = mg(z + R) = mg(R\cos{\theta} + R),
\end{equation}

\begin{equation}
 \therefore V = mgR(\cos{\theta} + 1).
\end{equation}

La energía cinética del sistema se escribe como 

\begin{equation}
 T = \frac{m}{2}(\dot{x}^2 + \dot{y}^2 + \dot{z}^2),
\end{equation}

debido a que existe la constricción de que $R = \text{cte}$, y recordamos que 
estamos usando coordenadas esféricas, tenemos que 

\begin{align*}
 \dot{x} = R\dot{\theta}\cos{\theta}\cos{\phi} - R\dot{\phi}\sen{\theta}\sen{\phi}, \\
 \dot{y} = R\dot{\theta}\cos{\theta}\sen{\phi} + R\dot{\phi}\sen{\theta}\cos{\phi}, \\
 \dot{z} = - R\dot{\theta}\sen{\theta}.
\end{align*}

Entonces la energía cinética se transforma en (usando el hecho de que la velocidad 
angular es constante y por lo tanto $\dot{\phi} = \omega$),

\begin{equation}
 T = \frac{1}{2}m\left[R\dot{\theta} + R^2\omega^2\sen^2{\theta} \right],
\end{equation}

por lo que la lagrangiana del sistema será

\begin{equation}
 L = T - V = \frac{1}{2}m\left[R\dot{\theta} + R^2\omega^2\sen^2{\theta} \right] 
 -  mgR(\cos{\theta} + 1).
\end{equation}

Y las ecuaciones de movimiento las obtenemos con la ecuación de Lagrange para 
$\theta$, 

\begin{equation}
 \frac{d}{dt}\left(\frac{\partial L}{\partial \dot{\theta}}\right) - 
 \frac{\partial L}{\partial \theta} = 0,
\end{equation}

\begin{equation}
 \frac{d}{dt}\left( m\dot{\theta}R^2 \right) - 
 - mR^2\omega^2\sen{\theta}\cos{\theta} + mgR\sen{\theta}  = 0,
\end{equation}

\begin{equation}
 \boxed{\therefore R\ddot{\theta} - R\omega^2\sen{\theta}\cos{\theta} + g\sen{\theta}
 = 0.}
\end{equation}

\begin{enumerate}[label=\alph*)]
\setcounter{enumi}{1}
 \item R:
\end{enumerate}

La energía total del sistema es 

\begin{equation}
 E = T + V =  \frac{1}{2}m\left[R\dot{\theta} + R^2\omega^2\sen^2{\theta} \right] 
 +  mgR(\cos{\theta} + 1),
\end{equation}

de donde vemos que podemos estudiar este sistema como uno con potencial efectivo 

\begin{equation}
 V_{ef} =  \frac{1}{2}mR^2\omega^2\sen^2{\theta}  +  mgR(\cos{\theta} + 1),
\end{equation}

para calcular $\omega_0$, debemos derivar $V_{ef}$ con respecto a $\theta$ e 
igualar a cero, con lo cual obtendremos el ángulo de equilibrio y con esto 
podemos saber en que rangos se encontrará la frecuencia solicitada.

\begin{equation}
 \frac{d}{d\theta} \left[\frac{1}{2}mR^2\omega^2\sen^2{\theta} + 
 mgR(\cos{\theta} + 1) \right] = 0,
\end{equation}

\begin{equation}
 \cancel{m}R^{\cancel{2}}\omega^2\cancel{\sen{\theta}}\cos{\theta} - \cancel{m}\cancel{R} 
 g\cancel{\sen{\theta}} = 0,
\end{equation}

\begin{equation}
 R\omega^2\cos{\theta} - g = 0,
\end{equation}

de donde vemos que 

\begin{equation}
 \theta = \arccos{\left[\frac{g}{R\omega^2}\right]},
\end{equation}

con lo cual, debido a la definición de $\arccos$ que 

\begin{equation}
 \frac{g}{R\omega^2} < 1,
\end{equation}

y entonces 

\begin{equation}
 \boxed{\omega_0 > \sqrt{\frac{g}{R}}.}
\end{equation}

\end{document}