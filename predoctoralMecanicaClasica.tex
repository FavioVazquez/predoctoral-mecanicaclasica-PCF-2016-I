\documentclass[a4paper,10pt]{article}
\usepackage[utf8]{inputenc}
\usepackage[spanish]{babel}
\usepackage[affil-it]{authblk}
\usepackage{enumerate}
\usepackage{graphicx}
\usepackage{hyperref}
\usepackage{amsmath}
\usepackage{amssymb}
\usepackage{cancel}
\usepackage[usenames, dvipsnames]{color}
\usepackage{tikz}
\usepackage[labelfont=bf]{caption}
\usepackage{subcaption} %Multiple images
\usepackage{multicol} % Multiple columns
\usepackage{float}
\usepackage{cleveref}
 \usepackage{relsize} % bigger math symbols
\usepackage[margin=1.4in]{geometry}
\usepackage[titletoc,toc,title]{appendix}
\usepackage{enumitem}
\usepackage{etoolbox}
\usepackage{mdframed} %frame theorems
\usetikzlibrary{calc}
\numberwithin{equation}{section}

% Enviroment for theorems
\newmdtheoremenv[frametitle=Teorema]{theo}{Theorem}

% Circled words
\newcommand{\circled}[2][]{%
  \tikz[baseline=(char.base)]{%
    \node[shape = circle, draw, inner sep = 1pt]
    (char) {\phantom{\ifblank{#1}{#2}{#1}}};%
    \node at (char.center) {\makebox[0pt][c]{#2}};}}
\robustify{\circled}

%Appendices in spanish
\renewcommand{\appendixname}{Ap\'endices}
\renewcommand{\appendixtocname}{Ap\'endices}
\renewcommand{\appendixpagename}{Ap\'endices}

%Zero delimiter
\newcommand{\zerodel}{.\kern-\nulldelimiterspace}

%Columns separation
\setlength{\columnsep}{1cm}

%Indentation
\setlength{\parindent}{0ex}

%Multiple References

\crefrangelabelformat{equation}{(#3#1#4--#5\crefstripprefix{#1}{#2}#6)}

\usepackage{xparse}

%Boxes

\newcommand*{\boxcolor}{blue}
\makeatletter
\renewcommand{\boxed}[1]{\textcolor{\boxcolor}{%
\tikz[baseline={([yshift=-1ex]current bounding box.center)}] \node [rectangle, minimum width=1ex,rounded corners,draw] {\normalcolor\m@th$\displaystyle#1$};}}
 \makeatother

%Constantes
\newcommand{\euler}{\mathrm{e}}
\newcommand{\im}{i}

%Lemas, teoremas, definiciones y pruebas
\newcommand{\definicion}{\textbf{Definición: }}
\newcommand{\lema}{\textbf{Lema: }}
\newcommand{\teorema}{\textbf{Teorema: }}
\newcommand{\prueba}{\textbf{Prueba: }}
\newcommand{\proposicion}{\textbf{Proposición: }}
\newcommand{\corolario}{\textbf{Corolario: }}

% Definición de las secciones y su numeración

\makeatletter
\def\@seccntformat#1{%
  \expandafter\ifx\csname c@#1\endcsname\c@section\else
  \csname the#1\endcsname\quad
  \fi}
\makeatother

%opening
\title{Examen Predoctoral de Mecánica Clásica. \\
Semestre 2016-I.}
\author{Favio Vázquez\thanks{Correo: favio.vazquezp@gmail.com}}\affil{Instituto de Ciencias Nucleares. Universidad Nacional Autónoma de México.}
\date{}

\begin{document}

\makeatletter
\def\@maketitle{%
  \newpage
  \null
  \vskip 2em%
  \begin{center}%
  \let \footnote \thanks
    {\Large\bfseries \@title \par}%
    \vskip 1.5em%
    {\normalsize
      \lineskip .5em%
      \begin{tabular}[t]{c}%
        \@author
      \end{tabular}\par}%
    \vskip 1em%
    {\normalsize \@date}%
  \end{center}%
  \par
  \vskip 1.5em}
\makeatother

\maketitle

\section{Preguntas teóricas}

\subsection{Pregunta teórica 1}

Sobre las formulaciones de la mecánica, discuta los siguientes puntos:

\begin{enumerate}[label=\alph*)]
 \item Bajo qué condiciones las ecuaciones de Euler-Lagrange determinan todas las aceleraciones del
 sistema.
 \item En el caso Hamiltoniano, cuál es la condición equivalente.
 \item Dentro del formalismo de Hamilton-Jacobi qué garantiza el que la funcional generadora
 proporcione una solución de las ecuaciones de Hamilton.
\end{enumerate}

\vspace{.3cm}

\underline{Solución:} \vspace{.3cm}

\subsection{Pregunta Teórica 2}

Considere un cuerpo rígido con momentos de inercia $I_1 > I_2 > I_3$. Si sobre el cuerpo no se 
ejercen torcas, ¿qué ejes del cuerpo son estables e inestables bajo pequeñas perturbaciones y 
por qué?

\vspace{.3cm}

\underline{Solución:} \vspace{.3cm} 

\subsection{Pregunta 3}

Sobre un piso sin fricción, una partícula puntual choca elásticamente con una mancuerna de dos 
modos diferentes mostrados en la figura (considere que las partículas de la mancuerna son también
puntuales y que la barra tiene masa despreciable). ¿Qué cantidades se conservan en cada caso? 
¿En qué caso la rapidez del centro de masa de la mancuerna, después dela colisión, es mayor?

\vspace{.3cm}

AGREGAR FIGURA

\vspace{.3cm}

\section{Problemas}

\subsection{Problema 1}

La interacción clásica entre dos átomos de un gas inerte, cada uno de masa $m$ está dada por el
potencial 

$$
V(r) = - \frac{2A}{r^6} + \frac{B}{r^12}
$$

con $A$ y $B$ constantes positivas y $r$ la separación entre los dos átomos, $r = |\overrightarrow{r_1} 
- \overrightarrow{r_2}|$.

\begin{enumerate}[label=\alph*)]
 \item Obtenga la hamiltoniana para el sistema de los dos átomos.
 \item Describa completamente el (los) estado(s) clásico(s) de energía mínima del presente 
 sistema.
 \item Si la energía es un poco mayor que la mínima, ¿Cuáles son las posibles frecuencias 
 del movimiento del sistema?
\end{enumerate}

\vspace{.3cm}

\underline{Solución:} \vspace{.3cm}

\subsection{Problema 2}

Considere el siguiente sistema 

$$
L = \frac{m}{2}\dot{q}^2 - af(t)q
$$

con $f(t)$ una función arbitraria del tiempo pero integrable.

\begin{enumerate}[label=\alph*)]
 \item Considerando que una simetría del sistema es aquella que deja invariantes las 
 ecuaciones de movimiento. ¿Existe alguna simetría asociada a este sistema? Si es así,
 calcule la cantidad conservada correspondiente usando el teorema de Noether.
 \item Muestre que, efectivamente su derivada total con respecto del tiempo es cero.
 \item Construya el Hamiltoniano del sistema y escriba la ecuación de Hamilton-Jacobi 
 correspondiente.
 \item Resuelva la ecuación de hamilton-Jacobi y encuentre la funcional generadora de 
 tipo 2.
 \item Considere que $f(t) = \exp{(-bt)}$ con $b > 0$ y las condiciones iniciales
 $q(0) = \beta$ y $\dot{q}(0) = \rho$. Usando la teoría de Hamilton-Jacobi, encuentre 
 la trayectoria de la partícula.
\end{enumerate}

\vspace{.3cm}

\underline{Solución:} \vspace{.3cm}

\subsection{Problema 3}

Una partícula de masa $m$ está restringida a moverse en el interior de un riel circular 
de radio $R$. El riel circular está fijado al piso en posición vertical. Un pequeño 
motor hacer girar el riel en torno al eje de simetría vertical con rapidez angular 
constante $\omega$ (ver figura). Considere el cero de energía potencial en el piso y 
sea $\theta$ el ángulo que forma el radio vector de posición de la partícula con el 
eje de rotación.

\begin{enumerate}[label=\alph*)]
 \item Determine el Lagrangiano del sistema con constricción y la ecuación de movimiento 
 en $\theta$ para la partícula.
 \item Para que exista una órbita a $\theta_{eq}= \text{cte}$ y distinta de cero, $\omega$
 tiene que ser mayor que cierta $\omega_0$. Determine $\omega_0$.
\end{enumerate}

\vspace{.3cm}

\underline{Solución:} \vspace{.3cm}

\end{document}