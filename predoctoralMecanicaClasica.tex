\documentclass[a4paper,10pt]{article}
\usepackage[utf8]{inputenc}
\usepackage[spanish]{babel}
\usepackage[affil-it]{authblk}
\usepackage{enumerate}
\usepackage{graphicx}
\usepackage{hyperref}
\usepackage{amsmath}
\usepackage{amssymb}
\usepackage{cancel}
\usepackage[usenames, dvipsnames]{color}
\usepackage{tikz}
\usepackage[labelfont=bf]{caption}
\usepackage{subcaption} %Multiple images
\usepackage{multicol} % Multiple columns
\usepackage{float}
\usepackage{cleveref}
 \usepackage{relsize} % bigger math symbols
\usepackage[margin=1.4in]{geometry}
\usepackage[titletoc,toc,title]{appendix}
\usepackage{enumitem}
\usepackage{etoolbox}
\usepackage{mdframed} %frame theorems
\usetikzlibrary{calc}
\numberwithin{equation}{section}

% Enviroment for theorems
\newmdtheoremenv[frametitle=Teorema]{theo}{Theorem}

% Circled words
\newcommand{\circled}[2][]{%
  \tikz[baseline=(char.base)]{%
    \node[shape = circle, draw, inner sep = 1pt]
    (char) {\phantom{\ifblank{#1}{#2}{#1}}};%
    \node at (char.center) {\makebox[0pt][c]{#2}};}}
\robustify{\circled}

%Appendices in spanish
\renewcommand{\appendixname}{Ap\'endices}
\renewcommand{\appendixtocname}{Ap\'endices}
\renewcommand{\appendixpagename}{Ap\'endices}

%Zero delimiter
\newcommand{\zerodel}{.\kern-\nulldelimiterspace}

%Columns separation
\setlength{\columnsep}{1cm}

%Indentation
\setlength{\parindent}{0ex}

%Multiple References

\crefrangelabelformat{equation}{(#3#1#4--#5\crefstripprefix{#1}{#2}#6)}

\usepackage{xparse}

%Boxes

\newcommand*{\boxcolor}{blue}
\makeatletter
\renewcommand{\boxed}[1]{\textcolor{\boxcolor}{%
\tikz[baseline={([yshift=-1ex]current bounding box.center)}] \node [rectangle, minimum width=1ex,rounded corners,draw] {\normalcolor\m@th$\displaystyle#1$};}}
 \makeatother

%Constantes
\newcommand{\euler}{\mathrm{e}}
\newcommand{\im}{i}

%Lemas, teoremas, definiciones y pruebas
\newcommand{\definicion}{\textbf{Definición: }}
\newcommand{\lema}{\textbf{Lema: }}
\newcommand{\teorema}{\textbf{Teorema: }}
\newcommand{\prueba}{\textbf{Prueba: }}
\newcommand{\proposicion}{\textbf{Proposición: }}
\newcommand{\corolario}{\textbf{Corolario: }}

% Definición de las secciones y su numeración

\makeatletter
\def\@seccntformat#1{%
  \expandafter\ifx\csname c@#1\endcsname\c@section\else
  \csname the#1\endcsname\quad
  \fi}
\makeatother

%opening
\title{Examen Predoctoral de Mecánica Clásica. \\
Semestre 2016-I.}
\author{Favio Vázquez\thanks{Correo: favio.vazquezp@gmail.com}}\affil{Instituto de Ciencias Nucleares. Universidad Nacional Autónoma de México.}
\date{}

\begin{document}

\makeatletter
\def\@maketitle{%
  \newpage
  \null
  \vskip 2em%
  \begin{center}%
  \let \footnote \thanks
    {\Large\bfseries \@title \par}%
    \vskip 1.5em%
    {\normalsize
      \lineskip .5em%
      \begin{tabular}[t]{c}%
        \@author
      \end{tabular}\par}%
    \vskip 1em%
    {\normalsize \@date}%
  \end{center}%
  \par
  \vskip 1.5em}
\makeatother

\maketitle

\section{Preguntas teóricas}

\subsection{Pregunta teórica 1}

Sobre las formulaciones de la mecánica, discuta los siguientes puntos:

\begin{enumerate}[label=\alph*)]
 \item Bajo qué condiciones las ecuaciones de Euler-Lagrange determinan todas las aceleraciones del
 sistema.
 \item En el caso Hamiltoniano, cuál es la condición equivalente.
 \item Dentro del formalismo de Hamilton-Jacobi qué garantiza el que la funcional generadora
 proporcione una solución de las ecuaciones de Hamilton.
\end{enumerate}

\vspace{.3cm}

\underline{Solución:} \vspace{.3cm}

\subsection{Pregunta Teórica 2}

Considere un cuerpo rígido con momentos de inercia $I_1 > I_2 > I_3$. Si sobre el cuerpo no se 
ejercen torcas, ¿qué ejes del cuerpo son estables e inestables bajo pequeñas perturbaciones y 
por qué?

\vspace{.3cm}

\underline{Solución:} \vspace{.3cm} 

Debemos ver en qué casos, con respecto a la dirección de la velocidad angular, el 
movimiento será estable y con respecto a qué eje principal. Partimos de las ecuaciones 
de Euler para un cuerpo rígido al cual no se le aplican torcas,

\begin{align}
 \label{eq:pregunta2eq1}
 I_1\dot{\omega}_1 - \omega_2\omega_3(I_2 - I_3) = 0, \\
 \label{eq:pregunta2eq2}
 I_2\dot{\omega}_2 - \omega_3\omega_1(I_3 - I_1) = 0, \\
 \label{eq:pregunta2eq3}
 I_3\dot{\omega}_3 - \omega_1\omega_2(I_1 - I_2) = 0.
\end{align}

Estas ecuaciones nos permitirán estudiar, cuando el cuerpo esté en movimiento, qué 
condiciones deben cumplirse para que el movimiento sea estable.

\vspace{.3cm}

Si suponemos que la velocidad angular es en dirección al eje $x$, entonces se 
cumplirá que $\omega_1 \gg \omega_2,\omega_3$. Ahora debido a que $\omega_2$ y 
$\omega_3$ se mantendrán pequeños con respecto a $\omega_1$, el movimiento será 
estable y debido a que no hay torque $|\overrightarrow{\omega}| = \text{cte}$, y 
debido a que $\omega = \sqrt{\omega_1^2 + \omega_2^2 + \omega_3^2}$, tendremos que 
$\omega_2^2 + \omega_3^2 \ll \omega_1^2$, y entonces $\omega = \sqrt{\omega_1^2} = 
\omega_1$, y entonces podemos tomar a $\omega_1$ como constante, al menos 
a primer orden.

Tomando la derivada temporal de \eqref{eq:pregunta2eq2} 

\begin{equation}
 \frac{d}{dt}\left[I_2\dot{\omega}_2 - \omega_3\omega_1(I_3 - I_1) \right] = 0,
\end{equation}

y debido a que $\omega_1$ es constante, 

\begin{equation}
 I_2\ddot{\omega}_2 - \dot{\omega_3}\omega_1(I_3 - I_1) = 0,
\end{equation}

\begin{equation}
 \therefore \ddot{\omega}_2 = \frac{\dot{\omega}_3\omega_1(I_3-I_1)}{I_2}.
  \label{eq:pregunta2eq4}
\end{equation}

Sustituyendo $\dot{\omega}_3$ de \eqref{eq:pregunta2eq3}, 

\begin{equation}
 \ddot{\omega_2} = \frac{\omega_2\omega_1^2(I_3 - I_1)(I_1 - I_2)}{I_2I_3}.
 \label{eq:pregunta2eq5}
\end{equation}

Tomando la derivada temporal de \eqref{eq:pregunta2eq3} 

\begin{equation}
 \frac{d}{dt}\left[I_3\dot{\omega}_3 - \omega_1\omega_2(I_1 - I_2) \right] = 0,
\end{equation}

\begin{equation}
 I_3\ddot{\omega}_3 - dot{\omega}_2\omega_1(I_1 - I_2) = 0,
\end{equation}

\begin{equation}
 \therefore \ddot{\omega}_3 = \frac{\dot{\omega}_2\omega_1(I_1-I_2)}{I_3}.
 \label{eq:pregunta2eq6}
\end{equation}

Sustituyendo $\dot{\omega}_2$ de \eqref{eq:pregunta2eq2}, en \eqref{eq:pregunta2eq6}

\begin{equation}
 \ddot{\omega_3} = \frac{\omega_3\omega_1^2(I_1 - I_2)(I_3 - I_1)}{I_3I_2}.
 \label{eq:pregunta2eq7}
\end{equation}

Ahora como $I_1 > I_2,I_3$, los lados derechos de \eqref{eq:pregunta2eq5} y 
\eqref{eq:pregunta2eq7} serán negativos y por lo tanto tendremos un equilibrio 
estable, con un movimiento tipo oscilador armónico, y $\omega_2$ y $\omega_3$ 
oscilarán al rededor del punto de equilibrio. Vemos entonces que un movimiento 
en dirección del eje $x$ es estable, y por consiguiente el momento principal 
de inercia $I_1$ será estable. 

\vspace{.3cm}

Supongamos ahora que el movimiento es en dirección al eje $z$, utilizando los 
mismos argumentos, vemos que debido a que $\omega_3 \gg \omega_1,\omega_2$, 
$|\omega| = \text{cte}$, entonces $\omega = \omega_3$, y podemos considerar 
que $\omega_3$ es estable, al menos a primer orden. 

Tomando la derivada temporal de \eqref{eq:pregunta2eq1} 

\begin{equation}
 \frac{d}{dt}\left[I_1\dot{\omega}_1 - \omega_2\omega_3(I_2 - I_3) \right] = 0,
\end{equation}

y debido a que $\omega_3$ es constante, 

\begin{equation}
 I_1\ddot{\omega}_1 - \dot{\omega}_2\omega_3(I_2 - I_3) = 0,
\end{equation}

\begin{equation}
 \therefore \ddot{\omega}_1 = \frac{\dot{\omega}_2\omega_3(I_2-I_3)}{I_1}.
  \label{eq:pregunta2eq8}
\end{equation}

Sustituyendo $\dot{\omega}_2$ de \eqref{eq:pregunta2eq2}, 

\begin{equation}
 \ddot{\omega_1} = \frac{\omega_1\omega_3^2(I_2 - I_3)(I_3 - I_1)}{I_2I_1}.
 \label{eq:pregunta2eq9}
\end{equation}

Tomando la derivada temporal de \eqref{eq:pregunta2eq2} 

\begin{equation}
 \frac{d}{dt}\left[I_2\dot{\omega}_2 - \omega_3\omega_1(I_3 - I_1) \right] = 0,
\end{equation}

\begin{equation}
 I_2\ddot{\omega}_2 - \dot{\omega}_1\omega_3(I_3 - I_1) = 0,
\end{equation}

\begin{equation}
 \therefore \ddot{\omega}_3 = \frac{\dot{\omega}_1\omega_3(I_3-I_1)}{I_2}.
 \label{eq:pregunta2eq10}
\end{equation}

Sustituyendo $\dot{\omega}_1$ de \eqref{eq:pregunta2eq1}, en \eqref{eq:pregunta2eq10}

\begin{equation}
 \ddot{\omega_2} = \frac{\omega_2\omega_3^2(I_3 - I_1)(I_2- I_3)}{I_1I_2}.
 \label{eq:pregunta2eq11}
\end{equation}

De nuevo como $I_1 > I_2,I_3$, los lados derechos de \eqref{eq:pregunta2eq9} y 
\eqref{eq:pregunta2eq11} serán negativos y por lo tanto tendremos un equilibrio 
estable, con un movimiento tipo oscilador armónico, y $\omega_1$ y $\omega_2$ 
oscilarán al rededor del punto de equilibrio. Vemos entonces que un movimiento 
en dirección del eje $z$ es estable, y por consiguiente el momento principal 
de inercia $I_3$ será estable. 

\vspace{.3cm}

Por último, supongamos que el movimiento es en dirección al eje $y$,  y utilizando los 
mismos argumentos, vemos que debido a que $\omega_2 \gg \omega_1,\omega_3$, 
$|\omega| = \text{cte}$, entonces $\omega = \omega_2$, y podemos considerar 
que $\omega_2$ es estable, al menos a primer orden. 

Tomando la derivada temporal de \eqref{eq:pregunta2eq1} 

\begin{equation}
 \frac{d}{dt}\left[I_1\dot{\omega}_1 - \omega_2\omega_3(I_2 - I_3) \right] = 0,
\end{equation}

y debido a que $\omega_2$ es constante, 

\begin{equation}
 I_1\ddot{\omega}_1 - \dot{\omega}_3\omega_2(I_2 - I_3) = 0,
\end{equation}

\begin{equation}
 \therefore \ddot{\omega}_1 = \frac{\dot{\omega}_3\omega_2(I_2-I_3)}{I_1}.
  \label{eq:pregunta2eq12}
\end{equation}

Sustituyendo $\dot{\omega}_3$ de \eqref{eq:pregunta2eq3}, 

\begin{equation}
 \ddot{\omega_1} = \frac{\omega_1\omega_2^2(I_1 - I_2)(I_2 - I_3)}{I_3I_1}.
 \label{eq:pregunta2eq13}
\end{equation}

Tomando la derivada temporal de \eqref{eq:pregunta2eq3} 

\begin{equation}
 \frac{d}{dt}\left[I_3\dot{\omega}_3 - \omega_1\omega_2(I_1 - I_2) \right] = 0,
\end{equation}

\begin{equation}
 I_3\ddot{\omega}_3 - \dot{\omega}_1\omega_2(I_1 - I_2) = 0,
\end{equation}

\begin{equation}
 \therefore \ddot{\omega}_3 = \frac{\dot{\omega}_1\omega_2(I_1-I_2)}{I_3}.
 \label{eq:pregunta2eq14}
\end{equation}

Sustituyendo $\dot{\omega}_1$ de \eqref{eq:pregunta2eq1}, en \eqref{eq:pregunta2eq14}

\begin{equation}
 \ddot{\omega_3} = \frac{\omega_3\omega_2^2(I_1 - I_2)(I_2- I_3)}{I_3I_2}.
 \label{eq:pregunta2eq15}
\end{equation}

Como $I_2 > I_3$ y $I_1 > I_2$, los lados derechos de \eqref{eq:pregunta2eq13} y 
\eqref{eq:pregunta2eq15} serán positivos y por lo tanto tendremos un equilibrio 
inestable. Vemos entonces que un movimiento en dirección del eje $y$ es inestable, 
y por consiguiente el momento principal de inercia $I_2$ será estable. 

\vspace{.3cm}

Entonces concluimos que los momentos de inercia estables son el mayor $I_1$ y 
el menor $I_3$, siendo $I_2$ el intermedio, inestable. Esto puede verse fácilmente 
si intentamos lanzar una raqueta al aire al mismo tiempo que le imprimimos una 
rotación, ya sea en torno al eje definido por el mango, al eje perpendicular a la 
pala o al perpendicular al mango contenido en la pala; en los primeros casos es 
fácil hacerlo sin producir fuertes bamboleos, en el tercero es prácticamente 
imposible (que es el eje correspondiente al momento principal de inercia intermedio).

\subsection{Pregunta 3}

Sobre un piso sin fricción, una partícula puntual choca elásticamente con una mancuerna de dos 
modos diferentes mostrados en la figura (considere que las partículas de la mancuerna son también
puntuales y que la barra tiene masa despreciable). ¿Qué cantidades se conservan en cada caso? 
¿En qué caso la rapidez del centro de masa de la mancuerna, después dela colisión, es mayor?

\vspace{.3cm}

AGREGAR FIGURA

\vspace{.3cm}

\section{Problemas}

\subsection{Problema 1}

La interacción clásica entre dos átomos de un gas inerte, cada uno de masa $m$ está dada por el
potencial 

$$
V(r) = - \frac{2A}{r^6} + \frac{B}{r^12}
$$

con $A$ y $B$ constantes positivas y $r$ la separación entre los dos átomos, $r = |\overrightarrow{r_1} 
- \overrightarrow{r_2}|$.

\begin{enumerate}[label=\alph*)]
 \item Obtenga la hamiltoniana para el sistema de los dos átomos.
 \item Describa completamente el (los) estado(s) clásico(s) de energía mínima del presente 
 sistema.
 \item Si la energía es un poco mayor que la mínima, ¿Cuáles son las posibles frecuencias 
 del movimiento del sistema?
\end{enumerate}

\vspace{.3cm}

\underline{Solución:} \vspace{.3cm}

\subsection{Problema 2}

Considere el siguiente sistema 

$$
L = \frac{m}{2}\dot{q}^2 - af(t)q
$$

con $f(t)$ una función arbitraria del tiempo pero integrable.

\begin{enumerate}[label=\alph*)]
 \item Considerando que una simetría del sistema es aquella que deja invariantes las 
 ecuaciones de movimiento. ¿Existe alguna simetría asociada a este sistema? Si es así,
 calcule la cantidad conservada correspondiente usando el teorema de Noether.
 \item Muestre que, efectivamente su derivada total con respecto del tiempo es cero.
 \item Construya el Hamiltoniano del sistema y escriba la ecuación de Hamilton-Jacobi 
 correspondiente.
 \item Resuelva la ecuación de hamilton-Jacobi y encuentre la funcional generadora de 
 tipo 2.
 \item Considere que $f(t) = \exp{(-bt)}$ con $b > 0$ y las condiciones iniciales
 $q(0) = \beta$ y $\dot{q}(0) = \rho$. Usando la teoría de Hamilton-Jacobi, encuentre 
 la trayectoria de la partícula.
\end{enumerate}

\vspace{.3cm}

\underline{Solución:} \vspace{.3cm}

\subsection{Problema 3}

Una partícula de masa $m$ está restringida a moverse en el interior de un riel circular 
de radio $R$. El riel circular está fijado al piso en posición vertical. Un pequeño 
motor hacer girar el riel en torno al eje de simetría vertical con rapidez angular 
constante $\omega$ (ver figura). Considere el cero de energía potencial en el piso y 
sea $\theta$ el ángulo que forma el radio vector de posición de la partícula con el 
eje de rotación.

\begin{enumerate}[label=\alph*)]
 \item Determine el Lagrangiano del sistema con constricción y la ecuación de movimiento 
 en $\theta$ para la partícula.
 \item Para que exista una órbita a $\theta_{eq}= \text{cte}$ y distinta de cero, $\omega$
 tiene que ser mayor que cierta $\omega_0$. Determine $\omega_0$.
\end{enumerate}

\vspace{.3cm}

\underline{Solución:} \vspace{.3cm}

\begin{enumerate}[label=\alph*)]
 \item R:
\end{enumerate}

Debido a que la única fuerza que existe es la gravitacional, el potencial se 
escribiría como (utilizando coordenadas polares)

\begin{equation}
 V = mgz = mgR\cos{\theta},
\end{equation}

ahora debido a que se solicita que el cero de la energía potencial esté en el piso, 
hay que cambiar el punto de referencia para el potencial, y ya que al potencial 
gravitacional solo le importa la diferencia entre el punto de referencia y el suelo, 
para acomodar esta condición hacemos primero,

\begin{equation}
 V = mg(z - z_0),
\end{equation}

donde el $z_0$ en este caso será $-R$ ya que el eje $z$ se ha tomando como positivo 
hacia arriba, entonces 

\begin{equation}
 V = mg(z + R) = mg(R\cos{\theta} + R),
\end{equation}

\begin{equation}
 \therefore V = mgR(\cos{\theta} + 1).
\end{equation}

La energía cinética del sistema se escribe como 

\begin{equation}
 T = \frac{m}{2}(\dot{x}^2 + \dot{y}^2 + \dot{z}^2),
\end{equation}

debido a que existe la constricción de que $R = \text{cte}$, y recordamos que 
estamos usando coordenadas esféricas, tenemos que 

\begin{align*}
 \dot{x} = R\dot{\theta}\cos{\theta}\cos{\phi} - R\dot{\phi}\sen{\theta}\sen{\phi}, \\
 \dot{y} = R\dot{\theta}\cos{\theta}\sen{\phi} + R\dot{\phi}\sen{\theta}\cos{\phi}, \\
 \dot{z} = - R\dot{\theta}\sen{\theta}.
\end{align*}

Entonces la energía cinética se transforma en (usando el hecho de que la velocidad 
angular es constante y por lo tanto $\dot{\phi} = \omega$),

\begin{equation}
 T = \frac{1}{2}m\left[R\dot{\theta} + R^2\omega^2\sen^2{\theta} \right],
\end{equation}

por lo que la lagrangiana del sistema será

\begin{equation}
 L = T - V = \frac{1}{2}m\left[R\dot{\theta} + R^2\omega^2\sen^2{\theta} \right] 
 -  mgR(\cos{\theta} + 1).
\end{equation}

Y las ecuaciones de movimiento las obtenemos con la ecuación de Lagrange para 
$\theta$, 

\begin{equation}
 \frac{d}{dt}\left(\frac{\partial L}{\partial \dot{\theta}}\right) - 
 \frac{\partial L}{\partial \theta} = 0,
\end{equation}

\begin{equation}
 \frac{d}{dt}\left( m\dot{\theta}R^2 \right) - 
 - mR^2\omega^2\sen{\theta}\cos{\theta} + mgR\sen{\theta}  = 0,
\end{equation}

\begin{equation}
 \boxed{\therefore R\ddot{\theta} - R\omega^2\sen{\theta}\cos{\theta} + g\sen{\theta}
 = 0.}
\end{equation}

\begin{enumerate}[label=\alph*)]
\setcounter{enumi}{1}
 \item R:
\end{enumerate}

La energía total del sistema es 

\begin{equation}
 E = T + V =  \frac{1}{2}m\left[R\dot{\theta} + R^2\omega^2\sen^2{\theta} \right] 
 +  mgR(\cos{\theta} + 1),
\end{equation}

de donde vemos que podemos estudiar este sistema como uno con potencial efectivo 

\begin{equation}
 V_{ef} =  \frac{1}{2}mR^2\omega^2\sen^2{\theta}  +  mgR(\cos{\theta} + 1),
\end{equation}

para calcular $\omega_0$, debemos derivar $V_{ef}$ con respecto a $\theta$ e 
igualar a cero, con lo cual obtendremos el ángulo de equilibrio y con esto 
podemos saber en que rangos se encontrará la frecuencia solicitada.

\begin{equation}
 \frac{d}{d\theta} \left[\frac{1}{2}mR^2\omega^2\sen^2{\theta} + 
 mgR(\cos{\theta} + 1) \right] = 0,
\end{equation}

\begin{equation}
 \cancel{m}R^{\cancel{2}}\omega^2\cancel{\sen{\theta}}\cos{\theta} - \cancel{m}\cancel{R} 
 g\cancel{\sen{\theta}} = 0,
\end{equation}

\begin{equation}
 R\omega^2\cos{\theta} - g = 0,
\end{equation}

de donde vemos que 

\begin{equation}
 \theta = \arccos{\left[\frac{g}{R\omega^2}\right]},
\end{equation}

con lo cual, debido a la definición de $\arccos$ que 

\begin{equation}
 \frac{g}{R\omega^2} < 1,
\end{equation}

y entonces 

\begin{equation}
 \boxed{\omega_0 > \sqrt{\frac{g}{R}}.}
\end{equation}




















\end{document}